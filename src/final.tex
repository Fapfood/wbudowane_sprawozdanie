\section{Ostatnie problemy}\label{sec:final_introduction}

Po upewnieniu się, że kompresja stosowana w formacie jpg jest bezkonkurencyjna,
oraz zbadaniu, jak w odpowiedni sposób możemy pobierać lokalizację stacji kosmicznej,
musieliśmy zmierzyć się z ostatnim problemem. Chcielismy dane o czasie wykonania
zdjęcia i położeniu stacji zapisywać w nazwie zdjęcia, jednak ESA poinformowało
nas, że nazwy plików mogą zostać zmienione w trakcie pobierania przez nich danych.
Musieliśmy wymyślić sposób, by móc jednoznacznie przyporządkować zdjęcie do daty
i lokalizacji.

Wymyśliliśmy trzy sposoby rozwiązania tego problemu:
\begin{itemize}
    \item zapisanie danych w metadanych zdjęcia - odrzuciliśmy to rozwiązanie, ponieważ
    ESA zmieniając nazwę pliku, mogła zmieniać również metadane - rozwiązanie było niepewne
    \item zapisanie danych na zdjęciu - odrzucone, ponieważ po pierwsze mogliśmy zastąpić
    potencjalnie ważny fragment zdjęcia, a po drugie odzyskanie tych danych mogłoby być
    trudne algorytmicznie
    \item haszowanie zdjęcia i zapis do pliku csv wartości hash, daty i lokalizacji -
    zdecydowaliśmy się na to rozwiązanie
\end{itemize}

\section{I ich rozwiązania}\label{sec:final_conclusion}

Musieliśmy także rozważyć, jak często wykonywać zdjęcia, by zająć jak najmniej miejsca na
karcie pamięci, a jednocześnie nawet po odrzuceniu niewyraźnych zdjęć mieć sfotografowany
cały możliwy obszar. Obliczyliśmy, że jeden piksel zdjęcia odpowiada 161 metrom na Ziemi,
a ISS porusza się z prędkością odpowiadającą przesunięciu o połowę widocznego na zdjęciu
obszaru w ciągu pół minuty. Ostatecznie zdecydowaliśmy się na wykonywanie zdjęć co 15 sekund.

Obecnie, po wielu próbach i błędach czekamy na wyniki działania naszego kodu na Międzynarodowej
Stacji Kosmicznej i przygotowujemy algorytmy uczenia maszynowego, które wykorzystamy do analizy
zebranych zdjęć.
