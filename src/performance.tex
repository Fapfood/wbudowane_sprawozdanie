\section{Opis problemu}\label{sec:performance_introduction}

Naszym pierwszym zamierzeniem było dokonywanie na pokładzie stacji zarówno pomiarów
(wykonania zdjęć), jak również ich analizy (przy pomocy algorytmów uczenia maszynowego). Jednak już po
rozpoczęciu przygotowywania programu dowiedzieliśmy się o przeciwskazaniach do takiego
podejścia do problemu.

Zostaliśmy poinformowani, że na pokładzie ISS nie znajduje RaspberryPi 3
Model B, jak początkowo sądziliśmy, a znacznie wolniejszy Raspberry Pi 1 Model B+. W związku
z takim obrotem spraw postanowiliśmy dokonać pomiarów prędkości tych dwóch modeli, jak
również komputera stacjonarnego PC, w celu sprawdzenia, czy analiza zdjęć na pokładzie będzie
wydajnościowo możliwa.

ESA udostępniła zbiór zdjęć wykonanych przez Raspberry w poprzedniej edycji eksperymentu, jednak
wśród nich nie znajdowały się zdjęcia wykonane po ciemnej stronie globu. Było to kolejne przeciwskazanie
<<<<<<< HEAD
do pierwotnego pomysłu, ponieważ musielibysmy spreparować dane testowe wykorzystując zdjęcia znalezione
=======
do pierwotnego pomysłu, ponieważ musielibysmy spreparować dane testowe wykorzystując zdjecia znalezione
>>>>>>> 9ad73dccd16b752cfc178a8f0fe140c3cb8feff4
w Internecie i wykonane przez inne urządzenia.

Nie mieliśmy wpływu na drugi problem, ale mogliśmy sprawdzić jak poważnym przeciwskazaniem jest pierwszy.
Postanowiliśmy zbadać wydajność modelu RaspberryPi 1 B+ i porównać ją z wydajnością modelu Raspberry 3 B oraz
komputerem stacjonarnym. Jako algorytm testowy postanowiliśmy wykorzystać metodę regresji logistycznej
wytrenowaną na zbiorze danych MNIST, ponieważ dane te są dostępne za darmo, a danych właściwe dla naszego
eksperymentu wykonamy dopiero w kolejnej jego fazie. Poniżej przedstawiamy wyniki czasowe ogólne jednego
przebiegu programu, oraz rozbite na zrobienie zdjęcia oraz jego analizę.

Próbą testową, którą kamera rejestrowała był zawsze obraz cyfry 8. Niestety prosty model liniowy jakim jest
regresja logistyczna w obrazie cyfry 8 rozpoznawał wszystkie możliwe cyfry. W naszym przypadku nie był to
jednak problem, ponieważ nie liczyła się dla nas poprawność rozpoznania, a jedynie czas przeprowadzenia analizy.

\section{Wyniki}\label{sec:performance_results}

\begin{table}[H]
    \centering
    \begin{threeparttable}
        \caption{Komputer klasy PC}
        \begin{tabular}{|l|l|ccc|}
            \toprule
            \thead{Rozpoznana \\ cyfra} & & \thead{Zrobienie \\ zdjęcia} &
            \thead{Analiza \\ zdjęcia} & \thead{Pojedyńczy \\ przebieg} \\
            \midrule
            1 & Najkrótszy czas[s] & 0.000 & 0.006 & 0.010 \\
            & Średni czas[s] & 0.000 & 0.006 & 0.010 \\
            & Najdłuższy czas[s] & 0.001 & 0.016 & 0.020 \\
            \midrule
            7 & Najkrótszy czas[s] & 0.000 & 0.006 & 0.010 \\
            & Średni czas[s] & 0.000 & 0.007 & 0.011 \\
            & Najdłuższy czas[s] & 0.001 & 0.012 & 0.016 \\
            \midrule
            8 & Najkrótszy czas[s] & 0.000 & 0.006 & 0.010 \\
            & Średni czas[s] & 0.000 & 0.006 & 0.010 \\
            & Najdłuższy czas[s] & 0.001 & 0.007 & 0.012 \\
            \midrule
            6 & Najkrótszy czas[s] & 0.000 & 0.006 & 0.010 \\
            & Średni czas[s] & 0.000 & 0.006 & 0.010 \\
            & Najdłuższy czas[s] & 0.001 & 0.007 & 0.011 \\
            \bottomrule
        \end{tabular}
        \begin{tablenotes}
            \item Procesor IntelCore i7-4790 3.60 Ghz
            \item RAM 16 GB
        \end{tablenotes}
    \end{threeparttable}
\end{table}

\begin{table}[H]
    \centering
    \begin{threeparttable}
        \caption{Raspberry Pi 1 Model B+}
        \begin{tabular}{|l|l|ccc|}
            \toprule
            \thead{Rozpoznana \\ cyfra} & & \thead{Zrobienie \\ zdjęcia} &
            \thead{Analiza \\ zdjęcia} & \thead{Pojedyńczy \\ przebieg} \\
            \midrule
            0 & Najkrótszy czas[s] & 0.106 & 0.739 & 0.933 \\
            & Średni czas[s] & 1.038 & 0.746 & 1.875 \\
            & Najdłuższy czas[s] & 2.385 & 0.760 & 2.097 \\
            \midrule
            2 & Najkrótszy czas[s] & 0.105 & 0.741 & 0.920 \\
            & Średni czas[s] & 0.881 & 0.765 & 1.769 \\
            & Najdłuższy czas[s] & 4.503 & 0.849 & 5.334 \\
            \midrule
            4 & Najkrótszy czas[s] & 0.104 & 0.737 & 0.936 \\
            & Średni czas[s] & 0.563 & 0.746 & 1.454 \\
            & Najdłuższy czas[s] & 1.557 & 0.759 & 2.373 \\
            \midrule
            7 & Najkrótszy czas[s] & 0.102 & 0.740 & 0.920 \\
            & Średni czas[s] & 0.554 & 0.751 & 1.421 \\
            & Najdłuższy czas[s] & 2.313 & 0.763 & 3.135 \\
            \midrule
            8 & Najkrótszy czas[s] & 0.174 & 0.743 & 0.958 \\
            & Średni czas[s] & 0.505 & 0.754 & 1.562 \\
            & Najdłuższy czas[s] & 2.720 & 0.763 & 3.538 \\
            \bottomrule
        \end{tabular}
        \begin{tablenotes}
            \item Procesor Broadcom CoS BCM2835 700 Mhz
            \item RAM: 512 MB
        \end{tablenotes}
    \end{threeparttable}
\end{table}

\begin{table}[H]
    \centering
    \begin{threeparttable}
        \caption{Raspberry Pi 3 Model B}
        \begin{tabular}{|l|l|ccc|}
            \toprule
            \thead{Rozpoznana \\ cyfra} & & \thead{Zrobienie \\ zdjęcia} &
            \thead{Analiza \\ zdjęcia} & \thead{Pojedyńczy \\ przebieg} \\
            \midrule
            0 & Najkrótszy czas[s] & 0.014 & 0.073 & 0.106 \\
            & Średni czas[s] & 0.014 & 0.133 & 0.177 \\
            & Najdłuższy czas[s] & 0.015 & 0.144 & 0.191 \\
            \midrule
            9 & Najkrótszy czas[s] & 0.007 & 0.080 & 0.104 \\
            & Średni czas[s] & 0.014 & 0.132 & 0.177 \\
            & Najdłuższy czas[s] & 0.015 & 0.135 & 0.180 \\
            \midrule
            1 & Najkrótszy czas[s] & 0.007 & 0.072 & 0.097 \\
            & Średni czas[s] & 0.014 & 0.132 & 0.177 \\
            & Najdłuższy czas[s] & 0.025 & 0.186 & 0.230 \\
            \midrule
            3 & Najkrótszy czas[s] & 0.014 & 0.124 & 0.168 \\
            & Średni czas[s] & 0.014 & 0.132 & 0.177 \\
            & Najdłuższy czas[s] & 0.014 & 0.135 & 0.179 \\
            \midrule
            7 & Najkrótszy czas[s] & 0.014 & 0.080 & 0.125 \\
            & Średni czas[s] & 0.014 & 0.134 & 0.179 \\
            & Najdłuższy czas[s] & 0.025 & 0.148 & 0.217 \\
            \midrule
            4 & Najkrótszy czas[s] & 0.014 & 0.131 & 0.176 \\
            & Średni czas[s] & 0.014 & 0.132 & 0.176 \\
            & Najdłuższy czas[s] & 0.014 & 0.132 & 0.176 \\
            \midrule
            5 & Najkrótszy czas[s] & 0.015 & 0.069 & 0.104 \\
            & Średni czas[s] & 0.015 & 0.138 & 0.183 \\
            & Najdłuższy czas[s] & 0.023 & 0.141 & 0.195 \\
            \midrule
            6 & Najkrótszy czas[s] & 0.007 & 0.068 & 0.092 \\
            & Średni czas[s] & 0.012 & 0.216 & 0.251 \\
            & Najdłuższy czas[s] & 0.034 & 0.335 & 0.426 \\
            \midrule
            8 & Najkrótszy czas[s] & 0.007 & 0.072 & 0.097 \\
            & Średni czas[s] & 0.014 & 0.139 & 0.192 \\
            & Najdłuższy czas[s] & 0.015 & 0.146 & 0.194 \\
            \midrule
            2 & Najkrótszy czas[s] & 0.007 & 0.072 & 0.097 \\
            & Średni czas[s] & 0.016 & 0.163 & 0.227 \\
            & Najdłuższy czas[s] & 0.028 & 0.267 & 0.356 \\
            \bottomrule
        \end{tabular}
        \begin{tablenotes}
            \item CPU: 4x ARM Cortex-A53, 1.2GHz (Quad Core 1.2GHz Broadcom BCM2837 64bit CPU)
            \item RAM: 1GB LPDDR2 (900 MHz)
            \item Rozbieżności w przypadku 6 i 2 były bardzo duże, najpierw seria bardzo szybkich rozpoznań,
            później seria bardzo powolnych.
        \end{tablenotes}
    \end{threeparttable}
\end{table}

\section{Opracowanie wyników}\label{sec:performance_results_preparation}

\begin{table}[H]
    \centering
    \begin{threeparttable}
        \caption{Uśredniony czas wszystkich prób}
        \begin{tabular}{|lccc|}
            \toprule
            & Zrobienie zdjęcia & Analiza zdjęcia & Pojedyńczy przebieg \\
            \midrule
            RaspberryPi 1 & 0.708 & 0.752 & 1.616 \\
            RaspberryPi 3 & 0.014 & 0.132 & 0.176 \\
            PC & 0.000 & 0.006 & 0.010 \\
            \bottomrule
        \end{tabular}
    \end{threeparttable}
\end{table}

\begin{table}[H]
    \centering
    \begin{threeparttable}
        \caption{Ilorazy poszczególnych wyników}
        \begin{tabular}{|lccc|}
            \toprule
            & Zrobienie zdjęcia & Analiza zdjęcia & Pojedyńczy przebieg \\
            \midrule
            Iloraz Pi1/Pi3 & 50,5x & 5,5x & 9x \\
            Iloraz Pi3/PC & 14x & 22x & 17,5x \\
            Iloraz Pi1/PC & 708x & 125x & 161,5x \\
            \bottomrule
        \end{tabular}
    \end{threeparttable}
\end{table}

\large Dodatkowe zastrzeżenia odnośnie Raspberry Pi 1: \normalsize
\begin{itemize}
    \item bardzo wolne łącze ethernetowe
    \footnotesize (poczta Gmail otwierała się ok 2 min, dla porównania przy dokładnie
    tej samej konfiguracji pozostałego sprzętu, jak również dokładnie tym samym egzemplarzem systemu, na
    Raspberry Pi 3 trwało to 0,5s) \normalsize \\
    \item niewydolność pozostałych interfejsów wejścia/wyjścia
    \footnotesize (w trakcie robienia zdjęcia przy pomocy modułu kamery, obraz
    na ekranie podpiętym pod HDMI stawał się czarny i wracał dopiero gdy zdjęcie zostało zapisane przez
    program) \normalsize \\
\end{itemize}

\section{Wnioski}\label{sec:performance_conclusion}

Najprawdopodobniej wykonanie zdjęcia na Raspberry Pi 1 było tak mało wydajne z powodu
niewydolności interfejsów i mniejszej mocy obliczeniowej tego modelu.
Nie bez znaczenia jest też fakt, że model 1 posiada procesor jednordzeniowy, zaś model 3 - czterordzeniowy.
To wszystko tłumaczy 50x wolniejsze działanie w porównaniu z Raspberry Pi 3.

Analiza zdjęcia na wszystkich 3 obiektach testowych była najbardziej stała czasowo, a jako
że to był główny cel naszego eksperymentu, to te wyniki są dla nas najważniejsze. Raspberry Pi 1
jest średnio 5,5x wolniejsze na samej analizie zdjęcia, które w tym przypadku było wykonywane
metodą prostej regresji liniowej.
W przypadku bardziej złożonych obliczeniowo modeli uczenia maszynowego można spodziewać się, że
rozbieżności czasowe będą dużo bardziej widoczne, dlatego musimy zrezygnować z analizy zdjęć
na stacji kosmicznej przez niską moc obliczeniową sprzętu.

Analizy zdjęć dokonamy na komputerze PC, już po wykonaniu części eksperymentu mającej
miejsce na pokładzie ISS.

Zmiana założeń eksperymentu niosła za sobą, poza rozwiązaniem problemu z wydajnością, także negatywne
konsekwencje. Miejsce na karcie pamięci mikrokomputera na stacji ISS które może zająć każdy zespół jest
ograniczone, dlatego też w kolejnej części eksperymentu próbowaliśmy znaleźć sposób na redukcję rozmiaru
wykonanych zdjęć.
